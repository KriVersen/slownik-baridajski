\documentclass[10pt,a4paper]{article}

\usepackage[top=3.5cm,bottom=3.5cm,left=3.7cm,right=3.7cm,columnsep=30pt]{geometry}
\usepackage[T1]{fontenc}
\usepackage[polish]{babel}
\usepackage[utf8x]{inputenc}
\usepackage{lmodern}
\selectlanguage{polish}
\usepackage{tikz}
\usepackage{tikz-dependency}
\usepackage{amsmath}
\usepackage{graphicx}
\usepackage{gb4e}
\usepackage{verse}
\usepackage{tipa}
\usepackage{hyperref}
\usepackage{morefloats}
\usepackage{microtype}
\usepackage{multicol}
\setcounter{secnumdepth}{0}
\setlength{\parindent}{0cm}
\setlength{\parskip}{4mm}
\usepackage{fancyhdr}
\fancyhead[L]{\textsf{\rightmark}}
\fancyhead[R]{\textsf{\leftmark}}
\renewcommand{\headrulewidth}{1.4pt}
\fancyfoot[C]{\textbf{\textsf{\thepage}}}
\renewcommand{\footrulewidth}{1.4pt}
\pagestyle{fancy}

% macro definitions
\newcommand{\entry}[1]{«#1»}
\newcommand{\nentry}[2]{#1. \entry{#2}}
\newcommand{\noun}[4]{\markboth{#1}{#1}\textbf{#1} /\textipa{#2}/ \textsc{rzecz. #3} #4 \\ }
\newcommand{\expr}[3]{\markboth{#1}{#1}\textbf{#1} /\textipa{#2}/ \textsc{wyraż.} #3 \\ }
\newcommand{\ver}[4]{\markboth{#1}{#1}\textbf{#1} /\textipa{#2}/ \textsc{czas. #3.} #4 \\ }
\newcommand{\conj}[3]{\markboth{#1}{#1}\textbf{#1} /\textipa{#2}/ \textsc{spój.} #3 \\ }

\title{Słownik baridajsko-polski}
\author{Kristian Arped \\ Adrien Eryk Esjar Józef Alatriste}

\begin{document}
\maketitle
\newpage

\begin{multicols}{2}
    \section{A}
    \noun{acsi}{aksi}{VII}{\entry{oko}}
    \expr{adiẽ}{adiẽ}{\entry{do widzenia}}
    \noun{agnil}{agnil}{VII}{\entry{koło}}
    \noun{alta}{alta}{II}{\entry{wszystko}}
    \noun{anlis}{anlis}{VII}{\entry{zwykłość, pospolitość}}
    \noun{aryo}{arjo}{I}{\entry{król}}
    \noun{aryo·ssa}{arjos:a}{II}{\entry{królestwo}}
    \ver{atnálf}{atna:lf}{reg}{\entry{przybliżać, tłumaczyć}}
    \conj{aux}{auʃ}{\entry{również, też, także}}
    \noun{ayre}{ajre}{III}{\entry{kłos}}
    

\end{multicols}

\end{document}
